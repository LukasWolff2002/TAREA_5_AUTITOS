\documentclass{article}  % Define la clase del documento.

% Paquetes de idioma y codificación
\usepackage[utf8]{inputenc}
\usepackage[T1]{fontenc}
\usepackage[spanish]{babel}  % Ajusta el idioma del documento a español.
\usepackage{tabularx}  % Permite la creación de tablas con ancho ajustable.

% Paquete de geometría para configurar márgenes y tamaño de papel
\usepackage[letterpaper, margin=3cm]{geometry}

% Paquetes de tipografía
\usepackage{mathptmx}    % Usa Times New Roman como fuente.
\usepackage{microtype}   % Mejora la justificación del texto.

% Paquetes para manejo de colores y gráficos
\usepackage{xcolor}      % Define y utiliza colores.
\usepackage{graphicx}    % Permite la inserción de imágenes.
\usepackage{tikz}        % Creación de gráficos vectoriales.

% Configuración de enlaces y referencias cruzadas
\usepackage{hyperref}
\hypersetup{
    colorlinks   = true,
    linkcolor    = darkblue,
    citecolor    = black,
    filecolor    = blue,
    urlcolor     = blue
}

%\usepackage{etoolbox}  % Carga este paquete en el preámbulo para modificar entornos de bibliografia
%\patchcmd{\thebibliography}{\chapter*}{\part*}{}{}

\usepackage{media9} % Permite la inserción de multimedia.

% Paquetes para la mejora visual de tablas y figuras
\usepackage{booktabs}    % Para tablas de alta calidad.
\usepackage{float}       % Controla la posición de figuras y tablas.

% Paquete para la personalización de códigos fuente
\usepackage{listings}
\lstset{
    literate=
    {á}{{\'a}}1 {é}{{\'e}}1 {í}{{\'i}}1 {ó}{{\'o}}1 {ú}{{\'u}}1
    {Á}{{\'A}}1 {É}{{\'E}}1 {Í}{{\'I}}1 {Ó}{{\'O}}1 {Ú}{{\'U}}1
    {ñ}{{\~n}}1 {Ñ}{{\~N}}1 {ü}{{\"u}}1 {Ü}{{\"U}}1,
    backgroundcolor=\color{backcolour},
    commentstyle=\color{codegreen},
    keywordstyle=\color{codepurple},
    numberstyle=\tiny\color{codegray},
    stringstyle=\color{red},
    basicstyle=\ttfamily\small,
    breakatwhitespace=false,
    breaklines=true,
    captionpos=b,
    keepspaces=true,
    numbers=left,
    numbersep=5pt,
    showspaces=false,
    showstringspaces=false,
    showtabs=false,
    tabsize=2,
    language=TeX,
    morecomment=[l]\#,
    frame=single,
    rulecolor=\color{black}
}

% Definición de colores al estilo Visual Studio Code
\definecolor{darkblue}{rgb}{0.0, 0.0, 0.55}  % Enlaces
\definecolor{codegreen}{rgb}{0.25, 0.49, 0.48}  % Comentarios
\definecolor{codegray}{rgb}{0.5, 0.5, 0.5}  % Números y anotaciones
\definecolor{codepurple}{rgb}{0.58, 0, 0.82}  % Palabras clave
\definecolor{backcolour}{rgb}{0.95, 0.95, 0.92}  % Fondo de código

% Configuraciones de párrafo y matemáticas
\usepackage{amsmath}
\usepackage{parskip}    % Espaciado entre párrafos.
\usepackage{ragged2e}   % Justificación mejorada.

% Configuración de secciones y encabezados
\usepackage{titlesec}
\titleclass{\part}{top} % Make part like a class
\titleformat{\part}[display]
  {\normalfont\huge\bfseries\centering}{\thepart}{40pt}{\Huge}
\titlespacing*{\part}{0pt}{-60pt}{10pt}
\titleformat{\part}
  {\normalfont\huge\bfseries}{}{0pt}{}

% Asegúrate de usar esto para mantener el estilo en las páginas de las partes
\titleformat{\part}[display]
  {\normalfont\huge\bfseries}{}{0pt}{}
  [\thispagestyle{fancy}] % Aplica el estilo fancy a las páginas de las partes

% Configuración de encabezados y pies de página personalizados
\usepackage{fancyhdr}
\pagestyle{fancy}
\fancyhf{}
\fancyhead[L]{\raisebox{0.20cm}{\textbf{Sistemas de Transporte}}}
\fancyhead[R]{\raisebox{0.1cm}{\includegraphics[width=0.25\linewidth]{LOGO_UNIVERSIDAD.jpg}}}
\fancyhead[C]{\rule{\textwidth}{0.6pt}}
\fancyfoot[C]{\rule{\textwidth}{0.6pt}}
\fancyfoot[R]{\raisebox{-1.5\baselineskip}{\thepage}}
\renewcommand{\headrulewidth}{0pt}
\renewcommand{\footrulewidth}{0pt}

% Configuración avanzada de geometría
\geometry{
  top=3.5cm, % Aumenta el espacio en la parte superior para subir el encabezado
  bottom=2.5cm,
  headheight=2.5cm % Aumenta la altura del encabezado si es necesario
}

% Configuracion de bibliografia
\usepackage{natbib}
\bibliographystyle{unsrtnat}  % Puedes cambiarlo por `unsrtnat`, `abbrvnat`, etc.
\usepackage{tabularx} 
\begin{document}
%----------------------------------------------------------------------------------------
% PORTADA
%----------------------------------------------------------------------------------------
\begin{titlepage}%Inicio de la carátula, solo modificar los datos necesarios
\newcommand{\HRule}{\rule{\linewidth}{0.5mm}} 
\center 
%----------------------------------------------------------------------------------------
%	ENCABEZADO
%----------------------------------------------------------------------------------------
\includegraphics[width=10cm]{LOGO_UNIVERSIDAD.jpg}\\ % Si esta plantilla se copio correctamente, va a llevar la imagen del logo de la facultad.OBS: Es necesario incluir el paquete: graphicx
\vspace{3cm}
%----------------------------------------------------------------------------------------
%	SECCION DEL TITULO
%----------------------------------------------------------------------------------------
\HRule \\[0.4cm]
{ \huge \bfseries Tarea 5}\\[0.4cm] % Titulo del documento
{ \huge \bfseries Sistemas de Transporte}\\[0.4cm] % Titulo del documento
\HRule \\[1.5cm]
 \vspace{5.5cm}
%----------------------------------------------------------------------------------------
%	SECCION DEL AUTOR
%----------------------------------------------------------------------------------------
\begin{flushright}
    { \textbf{Profesor:}\\
    Rafael Delpiano\\
    \vspace{0.2cm}
    \textbf{Alumno:} \\
    Lukas Wolff Casanova\\
}
\end{flushright}
\vspace{2cm}
%----------------------------------------------------------------------------------------
%	SECCION DE LA FECHA
%----------------------------------------------------------------------------------------
{\large \textbf{\today}}\\[2cm] % El comando \today coloca la fecha del dia, y esto se actualiza con cada compilacion, en caso de querer tener una fecha estatica, reemplazar el \today por la fecha deseada
\end{titlepage}

\newpage
\thispagestyle{empty} % Deshabilita el número de página en la página del índice
\include{resumen}
%----------------------------------------------------------------------------------------
%  INDICE
%----------------------------------------------------------------------------------------
\newpage
%----------------------------------------------------------------------------------------
%ACÁ EMPIEZA EL INFORME
\setcounter{page}{1}
%----------------------------------------------------------------------------------------
\section{Equilibrio de Wardrop}

Para hacer el cálculo de equilibrio de Wardrop, se identifica el trayecto más barato sin costo alguno. Posteriormente, se calcula si, usando todo el flujo en tal nodo, sigue siendo el más barato. De no ser así, se hace un equilibrio entre nodos. El valor \( u \) en base al rut es \( 30/1000 = 0.03 \).
\\ \\
El desarrollo de las preguntas se encuentra en el \href{https://github.com/LukasWolff2002/TAREA_5_AUTITOS/blob/main/CODIGO/codigo_final.py}{siguiente codigo}.

\subsection{Pregunta 1}

Se identifica el tramo \( (h + 10) \) como el más barato. Aun así, es necesario usar ambos tramos.
\\ \\
Tramo costo \( h + 10 \) => \( f = 6.02 \) => \( c = 16.2 \).
\\ \\
Tramo costo \( 2h + 12 \) => \( f = 2.01 \) => \( c = 16.02 \).

\subsection{Pregunta 2}

Se identifica el tramo \( (h+10) \) como el más barato. Aun así, es necesario incluir el segundo tramo \( (2h+12) \) y el tercer tramo \( (h+15) \).
\\ \\
Tramo costo \( h + 10 \) => \( f = 5.61 \) => \( c = 15.61 \).
\\ \\
Tramo costo \( 2h + 12 \) => \( f = 1.82 \) => \( c = 15.61 \).
\\ \\
Tramo costo \( h + 15 \) => \( f = 0.61 \) => \( c = 15.61 \).

\subsection{Pregunta 3}

Se identifica el tramo \( (h+10) \) como el más barato. Aun así, es necesario incluir el segundo tramo \( (2h+12) \), pero no así el tercer tramo \( (h+15) \).
\\ \\
Tramo costo \( h + 10 \) => \( f = 3.53 \) => \( c = 13.35 \).
\\ \\
Tramo costo \( 2h + 12 \) => \( f = 0.67 \) => \( c = 13.35 \).
\\ \\
Tramo costo \( h + 15 \) => \( f = 0 \) => \( c = 15 \).

\subsection{Pregunta 4}

Para el flujo de \( AB \), se identifican dos posibles rutas: la primera \( (A \rightarrow 3 \rightarrow 1 \rightarrow B) \) y la segunda \( (A \rightarrow B) \). En la ruta de \( B \) hacia \( A \) solo se identifica una ruta \( (B \rightarrow 1 \rightarrow 3 \rightarrow A) \), donde se observa que el tramo \( 1 \rightarrow 3 \) es compartido por una ruta de ida y la ruta de vuelta.
\\ \\
Primera Ruta \( AB \) => \( f = 5.25 \) => \( c = 28.77 \).
\\ \\
Segunda Ruta \( AB \) => \( f = 14.77 \) => \( c = 28.77 \).
\\ \\
Primera Ruta \( BA \) => \( f = 10.03 \) => \( c = 46.09 \).

\subsection{Pregunta 5}

Para el flujo desde 0 hacia 1 se identifican 3 rutas posibles: la primera \( (0 \rightarrow A \rightarrow 1) \), la segunda \( (0 \rightarrow B \rightarrow 1) \) y la tercera \( (0 \rightarrow A \rightarrow B \rightarrow 1) \). Para el flujo desde 0 hacia 2 se identifican 3 rutas posibles: la primera \( (0 \rightarrow 2) \), la segunda \( (0 \rightarrow B \rightarrow 2) \) y la tercera \( (0 \rightarrow A \rightarrow B \rightarrow 2) \). Se puede notar que hay tramos compartidos en varias ocasiones, por lo tanto, es necesario generar un análisis en conjunto de toda la red, dando como resultado:
\\ \\
Primera Ruta \( 0 \rightarrow 1 \) => \( f = 76.36 \) => \( c = 17.27 \).
\\ \\
Segunda Ruta \( 0 \rightarrow 1 \) => \( f = 76.36 \) => \( c = 17.27 \).
\\ \\
Tercera Ruta \( 0 \rightarrow 1 \) => \( f = 47.57 \) => \( c = 17.27 \).
\\ \\
Primera Ruta \( 0 \rightarrow 2 \) => \( f = 111.07 \) => \( c = 13.11 \).
\\ \\
Segunda Ruta \( 0 \rightarrow 2 \) => \( f = 55.53 \) => \( c = 13.11 \).
\\ \\
Tercera Ruta \( 0 \rightarrow 2 \) => \( f = 33.69 \) => \( c = 13.11 \).

\end{document}
